\documentclass[10pt, a4paper]{proc}
\usepackage{ngerman, fancyhdr}
\setlength{\headheight}{14.0pt}

\newcommand{\Charname}{Flitwig-ni Su }
\newcommand{\Charnametitel}{San Flitwig-ni Su }
\newcommand{\Kurznametitel}{San Flitwig }
\newcommand{\Kurzname}{Flitwig }

\hyphenation {Flit-wig Ca-lu }
\markboth{Charakterbeschreibung}{\large {\Charname}}
\pagestyle{fancy} \setlength{\parindent}{0pt}
\title{ \Huge{\Charname }}
\author{Jan H. Kr{\"u}ger\\jan@janhkrueger.de}
\date{Bad Homburg den \todayn}
\begin{document}
\maketitle
%\fontfamily{cmss} \selectfont
Stets zerstreut und meist mit einer Tasse Kakao in der Hand ist
dieser Person, ein Wichtel von Fuss bis Nasenspitzte, nur schwer
anzusehen das er dennoch mit einer der einfallsreichsten Wesen der
Scherbe ist und sogar den Titel eines Kakaopflanzers tr{\"a}gt. Nun
ja, er ist nicht wirklich ein Wichtel. Seine Mutter war eine
Wichtelin, sein Vater ein Halbling was in zu einem
Wichtel-Halbling macht. Dennoch hat er die meisten {\"a}ussereren
Charakteristika eines Wichtels.
\\\\Selbst f{\"u}r einen Wichtel gilt \Charname noch als recht klein,
er ist gerade mal um die 54 "`Fliks"' gro{\ss}. Das entspricht in etwa
90 der menschlichen Zentimeter. Dabei hat er eine lange, gekr{\"u}mmte
Nase mit welcher er vor dem Verzehr immer erst l{\"a}nger an einer
Kakaopflanze schnuppert denn er ist {\"u}berzeugt davon das nicht nur
die Zubereitung allein sondern auch die Herkunft der Pflanze den
Geschmack und somit den Genu{\ss} beim Verzehr beeinflusst.\\Dar{\"u}ber
hinaus wird bei der Sippe welcher er angeh{\"o}rt der Reichtum nicht
in Bohnen oder Kakaopflanzen gemessen sondern darin wieviele
Bohnen bei der Ernte in der Regel abfallen. Und dies ist, soviel
wei{\ss} \Kurzname, nicht nur von der Pflege sondern auch von der Art
der Kakaopflanze abh{\"a}ngig.\\\\Seine Haut hat eine Farbe von
Wiesenmoos angenommen und seine Augen sind durch den vielen
Kakaogenu{\ss} mittlerweile braun geworden. Kleiden tut er sich in f{\"u}r
Wichtel {\"u}bliche Sachen. Alle sind dabei aus Fasern der
Kakaopflanze gemacht, eine Kunst welche nur die "`Gr{\"u}nwichtel"'
beherrschen. Mittels intuitiver Magie und ein paar weiteren
Zutaten k{\"o}nnen sie die Fasern der Pflanze weich und dennoch stabil
und rei{\ss}fest machen.\\Mit seinen d{\"u}rren Gliedern und Fingern
greift er nach allem was ihm interessant erscheint und be{\"a}ugt es
genau um eventuell seine neueste Erfindung damit zu verbessern.
Dabei will er es nichtmal stehlen, solche Gedanken kommen ihm gar
nicht. Wenn er etwas interessantes sieht wird er nur stets von der
den Wichteln {\"u}blichen Neugier gepackt. Sobald ein Ding nicht mehr
sein interesse weckt l{\"a}sst er es auch in der Regel sofort wieder
fallen.\\Das andere mit seinem Vorgehen nicht einverstanden sein
k{\"o}nnten, daran hat er fr{\"u}her einmal schon gedacht doch diesen
Gedanken gleich als von den Sieben verlassen angesehen denn
niemand w{\"u}rde freiwillig auf das Leben verbessernde und
vereinfachende Erfindungen verzichten. Das w{\"u}rden nur Kakaodiebe
tun.\\\\Da er auf seinen Z{\"u}gen durch die Scherbe feststellte das
die Welt doch recht gef{\"a}hrlich sein kann, ja es sogar Monster gibt
welche M{\"a}uler aufweisen welche Wichtel komplett verschlingen
k{\"o}nnen und Z{\"a}hne l{\"a}nger wie eine Wichtelnase aufweisen, was in
etwa 12 Fliks sind, hat er sich zwei f{\"u}r ihn lebenswichtige Ger{\"a}te
gebastelt.\\Das eine ist eine "`Calu"' mit welcher \Kurzname
angespitzte {\"A}ste verschie{\ss}t. Trotz seiner gro{\ss}en {\"A}hnlichkeit mit
einem Bogen, um genau zu sein, es ist ein Bogen, besteht \Kurzname
auf Calu. Denn ein Bogen k{\"o}nnte niemals von einem Wichtel benutzt
werden. Wichtelexperimente mit Langb{\"o}gen f{\"u}hrten schon mehrmals
dazu das statt dem Pfeil der Wichtel flog und seine Kampfgef{\"a}hrten
lachend auf dem Boden lagen. Aus diesem Grunde ziehen es Wichtel
vor keine Langb{\"o}gen zu benutzen da sie mit einem hohem Eigenrisiko
verbunden sind.\\Seine zweite Erfindung benutzt immer dann wenn er
schnell fliehen muss oder hohe Stellen erreichen will. Im Grunde
ist es nichts weiter wie eine Armbrust welche ein Seil verschie{\ss}t
welches an einem Dreihacken befestigt ist. Danach zieht sich das
Seil von selbst wieder auf und \Kurzname, sofern er sich daran
festh{\"a}lt, mit hoch. So kann er B{\"a}ume und Felsen erreichen oder
schnell vor Monstern fliehen wo seine eigenen kleinen Beine nicht
ausreichen.\\Andererseits hilft ihm dieses Ger{\"a}t auch dabei aus
Kakaofeldern und Lagern zu fliehen sollte er entdeckt
werden...\\\\Dabei nutzt er den Kakao nicht nur um seinen Hunger
und Durst zu stillen. Er ist auch recht geschickt darin
Kakaosalben, Kakaocremes, Kakaoverb{\"a}nde und noch vieles weiteres
anzulegen. Er schlug sogar mal seinem Dorf vor die Haut von
Kakaobohnen mit Wachs zu f{\"u}llen welches echtem Kakao farblich
gleicht. Die so erhaltenen "`Wachsbohnen"' sollten dann an andere
Kakaoh{\"a}ndler eingetauscht werden um an echte Bohnen oder gar
richtige Pflanzen zu kommen. Pflanzen sind sehr wichtig f{\"u}r die
Gr{\"u}nwichtel denn mit ihnen k{\"o}nnen neue Plantagen angelegt werden.
Dieser Vorschlag wurde auch begeistert aufgenommen da es eine
allgemeine Erh{\"o}hung des Kakaoanteiles des ganzen Dorfes bedeutet
h{\"a}tte. Man hat ihn jedoch wieder fallen gelassen da man sich
eingestehen musste das man mit einer Wagenladung Kakaobohnen nicht
schnell genug verschwinden konnte bis der betrogene H{\"a}ndler auf
Wichteljagd gehen w{\"u}rde. Aber vergessen wurde der Vorschlag
nicht.\\\\Magie selbst ist f{\"u}r \Charname nicht ganz so wichtig. Er
kennt sich zwars ein wenig mit ihr aus, aber mehr wie die
Ver{\"a}nderung von Pflanzen, insbesondere das Beschleunigen des
Wachstums und allgemein der Ertragssteigerung, interessiert ihn
nicht wirklich. Sein Interesse, und auch der Grund warum ihm der
Titel Kakaopflanzer verliehen wurde, ist seine stetige
Erfindungsgabe. Sicher l{\"a}sst sich {\"u}ber den Sinn und Zweck so
mancher seiner Erfindungen streiten... aber wer will sich denn
schon mit einem Kakaopflanzer der Gr{\"u}nwichtel anlegen ? Noch dazu
einem welcher ohne Probleme zwischen den Beinen seines Feindes
durchlaufen kann um ihn von hinten anzugreifen ?\\Darin versteht
er sich, wie jeder Wichtel, auch hervorragend. Alleine wird
geflohen, in der Masse st{\"u}rzt man sich auf jeden Kakaodieb, ein
oder mehrere Wichtel krallen sich an jedem Arm und Bein sowie am
Rumpf und Kopf fest. Diese Taktik hat bisher noch jeden Kakaodieb
zu Fall gebracht, besonders eines wei{\ss} in \Kurzname in diesem
Zusammenhang: Es ist egal wie lang die Nase ist, hineinbei{\ss}en tut
jedem Weh. Auch die Ohren sind bei den anderen V{\"o}lkern gegen Bisse
sehr empfindlich. \Kurzname s eigene Ohren messen stolze sieben
und dreiviertel Fliks. Kein Wunder, er entstammt ja auch der Sippe
der "`Gro{\ss}lauschers"'.\\\\Von den Rotwichteln, mit welchen die
Gr{\"u}nwichtel entfernt verwandt sind, hat er noch keinen gesehen.
Zugegebenerma{\ss}en interessieren sie ihn auch nicht so wirklich,
viel mehr w{\"u}rde er gerne mal etwas Thaum probieren um zu verstehen
warum die Rotwichtel so sehr darauf versessen sind. Dies w{\"u}rde ihm
dabei helfen zu erkl{\"a}ren ob sich die Trennung der Wichtel in Gr{\"u}n-
und Rotwichtel aufgrund von st{\"a}rkerem Kakao bzw. Thaumvorkommen
ergeben hat. Denn wenn dem so ist, dann m{\"u}sste es seiner Meinung
nach auch irgendwo Blauwichtel geben welche den Vermutungen
zufolge nach Kaffee gieren oder auch Braunwichtel welche Tabak
konsumieren.\\Bei seinen Gr{\"u}nwichtelkollegen konnte \Charname
damit jedoch auf kein Geh{\"o}r sto{\ss}en denn sie k{\"o}nnen sich nicht
vorstellen wie man etwas anderes so lieben k{\"o}nnte wie
Kakao.\\\\Andererseits gelang es ihm eine Zeitlang jegliche
Kriminalit{\"a}t in seinem Dorf auszuschalten indem er den
Dorf{\"a}ltesten vorschlug Verbrechen mit bedingungslosem Kakaoentzug
zu ahnden. Da aber diejenigen welche mit Kakaoentzug bestraft
wurden anfingen heftige Entzugserscheinungen zu entwickeln und man
andererseits auch nicht grausam mit ihnen umgehen wollte hat sich
diese Form der Verbrechensvermeidung nicht durchsetzen
k{\"o}nnen.\\Ein anderer Vorschlag war jeden Verbrecher aufzuerlegen
eine Anzahl welche von der Schwere des Verbrechens abhing
wendariagef{\"a}llig neue Kakaopflanzen zum Wohle des Dorfes zu
pflanzen. Bei ganz schweren Verbrechen sollte dann sogar der
Verbrecher zum Wendariapriester geweiht werden um sich f{\"u}r den
Rest seines Lebens darum zu k{\"u}mmern das die Kakaopflanzen
Fruchtbar und stark bleiben. Das Dorf fand jedoch keinen Wichtel
welcher die l{\"a}ngere und vor allem disziplinierte Ausbildung zum
Priester durchhielt. Mal davon abgesehen das auch zwei Priester
mit der den Wichteln angeborenen Hektik nicht fertig wurden und
einer sogar den Wichtel aus dem Tempel warf weil dieser mehrmals
die Messe mit Zwischenrufen wie "`Schneller"' oder "`Lauter"'
st{\"o}rte zumal er auch in den Messen darauf bestand seinen Kakao zu
trinken... laut und f{\"u}r jeden h{\"o}rbar schl{\"u}rfend.\\\\Das soll aber
nicht hei{\ss}en das er nur eher unn{\"u}tze Dinge erfunden hat. Nein, es
waren auch n{\"u}tzliche Dinge dabei. So zum Beispiel sein sich selbst
schreibendes Gedichtbuch. Solange es Tinte hat schreibt es
fr{\"o}hlich vor sich hin. Nur hat sich bisher niemand gefunden der
den Dialekt des Buches lesen kann so das auch niemand wei{\ss} was
denn das Buch so die ganze Zeit {\"u}ber schreibt. Oder sein
selbstr{\"u}hrender Kochl{\"o}ffel. Er r{\"u}hrt perfekt. Nur auch zum
Beispiel im Rucksack oder gar wenn man ihn in der Hand
h{\"a}lt.\\\\Derzeit wandert er {\"u}ber die Scherbe um ein geeignetes
Pl{\"a}tzchen f{\"u}r ein zuk{\"u}nftiges Gasthaus zu suchen. Den Namen hat er
bereits: "`Zur Tasse des Heiligen Nes-Quik"'.
\end{document}
