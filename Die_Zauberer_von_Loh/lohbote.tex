\documentclass[10pt, a4paper, oneside]{book}
\usepackage{german, fancyhdr}
\setlength{\headheight}{14.0pt}
\newcommand{\autor}{Jan H. Kr{\"u}ger }
\newcommand{\email}{insulae@janhkrueger.de}

\markboth{Ausgabe 1 78.Bl{\"u}tenmond im Jahre 7}{\large {Loscher
Bote}} \pagestyle{fancy} \setlength{\parindent}{0pt}
\author{\autor\\ \email}
\date{Bad Homburg den \today}
\begin{document}
\fontfamily{cmss} \selectfont \tableofcontents
\chapter{Vorwort}
Ausgabe 1 im 78.Bl{\"u}tenmond des Jahres 7\\\\ Liebe Br{\"u}der, da dies
nun der erste Lohsche Bote ist - m{\"o}gen ihm noch viele weitere
folgen! - m{\"o}chte ich zuerst einige einleitende Worte
vorausschicken, die Natur dieser Zeitschrift betreffend. Der
Grundgedanke besteht darin, dass durch dieses w{\"o}chentlich
erscheinende Blatt die allgemeine Informationslage innerhalb der
Zauberer verbessert wird, da ein jeder {\"u}ber wichtige Geschehnisse,
intern und extern, unterrichtet wird. Neben der Tatsache, dass
somit die interne Kommunikation effektiver ablaufen wird, soll
damit auch das Interesse aller f{\"u}r die Geschehnisse am
Anschlagsbrett geweckt werden, die nicht nur die Wenigen angeht,
die sich dort des {\"O}fteren einfinden. Der Lohsche Bote soll aber
auch nicht nur ein Blatt f{\"u}r diese Nation sein, sondern auch von
dieser Nation, das bedeutet, dass ein jeder jederzeit die
M{\"o}glichkeit hat, hier seine neusten Forschungserkenntisse,
Gedichte oder Texte zu ver{\"o}ffentlichen, um sich mit anderen
Forschern/Schreibern auszutauschen. Des weiteren k{\"o}nnen Anzeigen
aufgegeben werden, die ich hier publizieren werde, um etwa auf
einen neuen Handelsposten aufmerksam zu machen oder {\"A}hnliches. Es
ist also durchaus erw{\"u}nscht, dass nicht nur meine eigenen
Nachrichten in diesem Blatte erscheinen, sondern auch die euren.
Ich m{\"o}chte an dieser Stelle auch noch einmal verdeutlichen, dass
der Lohsche Bote eine interne Zeitschrift der Zauberer von Loh ist
- jedwede Weitergabe an Au{\ss}enstehende (die Almaren ausgeschlossen)
ist untersagt, denn so manchen Wikinger w{\"u}rde es interessieren,
welcher Zauberer sich mit welchem streitet - und diese Genugtuung
wollen wir diesem Wikinger doch nicht geben. Viel Freude nun aber
mit der ersten Ausgabe des Lohschen Boten. Nai Anar caluva
tielyannar! Corintur Linyacelu\\Neben diesen Worten will ich noch
einige OOC-Worte nachschicken: Die Zeitung ist generell als
RPG-Sache gedacht, daher wird OOC derart mit rot gekennzeichnet.
Die Seite habe ich 1024*768 benutzend  gebaut, d.h. es kann bei
h{\"o}heren oder v.a. niedrigeren Aufl{\"o}sungen vorkommen, dass die
Sachen nicht mehr ganz passen; zu den Fonts: ich benutze f{\"u}r
{\"U}berschriften "`Old English Text"', das Vorwort "Times New Roman"
und die Artikel "`Lucida Calligraphy"', alles fonts, die
eigentlich bei Windows dabei sind, wenn es trotzdem Probleme gibt,
meldet euch und ich bastle das irgendwie um. Sowieso ist Kritik
und Feedback hoch erw{\"u}nscht - ich bin weder ein journalistisches
Genie noch ein begnadeter html-Coder, wenn ihr also
Verbesserungsvorschl{\"a}ge habt, dann immer her damit. Die URLs zu
den Ausgaben wird jeder auch weiterhin per diplo zugeschickt
bekommen, ich werde aber auch im Forum eine "Bibliothek" einf{\"u}gen,
wo die Links gesammelt werden. Abschlie{\ss}end will ich nochmal
wiederholen, dass ich mich {\"u}ber jeden Artikel eines anderen
Zauberers freue, egal, ob dieser {\"u}ber Metamagische Ph{\"a}nomene
theoretisiert, die Geschichte eines Charakters erz{\"a}hlt oder eine
Spendenbitte enth{\"a}lt. Das ist auch Eure Zeitschrift, nicht nur
meine! (Wenn mir zum Beispiel jemand sagen k{\"o}nnte wie ich den Text
dazu bringe gescheit um das Bild zu flie{\ss}en oder ob und wie man
den Text auch in Blocksatz schreiben kann, w{\"a}re ich demjenigen
sehr dankbar!) Jetzt will ich euch aber nicht noch weiter
aufhalten, also erstmal viel Spa{\ss} beim Lesen ;)

\chapter{Zwischenf{\"a}lle bei den Almaren} Wie im internen Anschlagsbrett nachzulesen ist, gab es in letzter
Zeit zwei F{\"a}lle von aggressivem Verhalten in der Nation Almaren,
deren erste Regel die Gewaltlosigkeit ist - Konflikte sollten {\"u}ber
uns, die Zauberer von Loh, gel{\"o}st werden. Einer der beiden "`{\"U}bel-
t{\"a}ter"', Merlin der Wei{\ss}e, wurde bereits aus der Nation verwiesen.
Er hatte Diebe ohne gen{\"u}gend Beweismaterial f{\"u}r ihre Vergehen
angegriffen, ja sogar im Zorne Mitbr{\"u}der beleidigt. Diesen recht
heftigen Versto{\ss} gegen die Regeln der Nation musste der Oberste
der Almaren, Claudius XI, ahnden, wenn auch mit Bedauern, ist
Merlin der Wei{\ss}e dennoch ein angenehmer Zeitgenosse,\\ wenn auch
f{\"u}r unsre Nation scheinbar ungeeignet. Der zweite Versto{\ss} gegen
diese wichtigste Regel unserer Ausbildungsnation war jedoch nicht
nicht derart extrem. Der weitbekannte Pl{\"u}nderer OgReXy war von
Elrond gemeuchelt worden, nachdem der R{\"a}uber dessen Geb{\"a}ude
wiederholt gepl{\"u}ndert hatte. Aufgrund dieser Umst{\"a}nde wurde durch
eine interne Abstimmung beschlossen, Elrond eine zweite Chance zu
geben. Und nicht nur das, vielmehr sollten, so Nationsf{\"u}hrer
Camaron, auch die Zauberer ihre "zweite Chance"nutzen, denn ihre
Pflicht w{\"a}re es gewesen die Anw{\"a}rter und Lehrlinge vor diesen
Pl{\"u}nderern und Dieben zu sch{\"u}tzen, auf dass f{\"u}r sie {\"u}berhaupt
keine Veranlassung entstehe, selbst Gewalt anwenden zu m{\"u}ssen.
Denn sollte bekannt werden, dass es unseren Sch{\"u}tzlingen bei den
Almaren untersagt ist, sich zu wehren, so werden sie sich vor
zwielichtigem Gesindel kaum retten k{\"o}nnen. Tatkr{\"a}ftige Handlungen
durch die Zauberer und eine bessere Kommunikation zwischen Mutter-
und Tochternation in solchen Dingen scheint n{\"o}tig. Daher werden
wir versuchen in jeder Ausgabe des Lohschen Boten Vorf{\"a}lle von
Barbarei, seien es Pl{\"u}nderungen, {\"U}berf{\"a}lle oder Diebst{\"a}hle, bei
denen entweder Almaren oder Zauberer betroffen sind, zu
ver{\"o}ffentlichen, auf dass jeder davon wisse und dementsprechend zu
handeln vermag. Eine erste Auflistung erscheint auch in diesem
Boten, unter Barbarei.\\\\\chapter{Glaubensfrage} Nicht nur dass
{\"u}berall auf der Scherbe Gewalt und Pl{\"u}nderungen st{\"a}rker um sich
greifen und Prophezeiungen vom Armageddon immer h{\"a}ufiger werden,
nun beginnt sich auch die Macht der sieben Unheiligen immer
offener zu regen. In den {\"o}ffentlichen Anschl{\"a}gen bekennen sich
mehr und mehr zu ihrem neuen Glauben an das B{\"o}se. Und nicht nur
das, an vielen Orten werden den G{\"o}tzen nun schon Tempel und
Kultst{\"a}tten errichtet und so manche Nation bekennt sich offen zu
einem der D{\"a}monen. Die Reaktion der Scherbenbev{\"o}lkerung darauf ist
zweigeteilt.\\\\Manche reagieren mit Toleranz und solch neuartigen
Worten wie "`Religionsfreiheit"', wollen die Diener der Unheiligen
nur nach Taten, nicht nach dem Glauben beurteilen und ihre Augen
fest verschlie{\ss}en vor dem, das die Unheiligen symbolisieren und
verbreiten. Andere wiederum, vor allem die Anh{\"a}nger Urvans,
predigen laut Hass und Zerst{\"o}rung der Ketzer und rufen mehrfach zu
Kreuzz{\"u}gen gegen die Kultst{\"a}tten auf, von denen sich einige nahe
dem Hort der Weisheit zu befinden scheinen. Der "`heilige Krieg"'
oder "`Glaubenskrieg"' ist in aller Munde, nur scheint sich noch
keine Front gegen die Ketzer gebildet zu haben und man mag sich
als Pherongl{\"a}ubiger durchaus fragen ob es denn recht ist, mit Hass
und Wut vorzugehen, sind doch ebengenau Eigenschaften der
Unheiligen, f{\"a}llt man so doch auch ihnen anheim, w{\"a}hrend man ihre
Kultst{\"a}tten niederbrennt. Ein Weg in der Mitte mus gefunden
werden, dar{\"u}ber ist man sich einig, doch wie dieser beschaffen
sein soll wird zur Zeit noch intern abgestimmt,die Stimme eines
jeden ist erbeten und erw{\"u}nscht, sowie eine Beteiligung an der
parallel standfindenden Diskussion. Die Frist f{\"u}r die Abstimmung
wurde auf drei Tage, also bis zum 3.Oktober, gesetzt.
International wurden zudem die Kirchenkonvente aller Sieben G{\"o}tter
einberufen, wo ein jeder, der zumindest den Rang eines Kardinals
inne hat, Stimm-und Sprachrecht hat. Noch sind die Konvente nicht
aktiv, doch einige Diskussionen werden bereits gef{\"u}hrt. Man wird
sehen, was die Zukunft bringt. M{\"o}ge Pheron uns durch die
Dunkelheit f{\"u}hren!

\chapter{Zweiter Schlussstrich unter den Fall Harsh Judgement} Die Stadt, die zuvor Roma hie{\ss} und das Hauptquartier der Terror-
nation "`Harsh Judgement"' war, wurde vor 10 Tagen von K{\"o}nig
Zaracas niedergebrannt. Damit wurde ein endg{\"u}ltiger Schluss-
strich unter die Angelegenheit gezogen. Dennoch wird niemand einen
derartigen Eintrag in den B{\"u}chern der Geschichtsschreiber finden,
denn der Zerst{\"o}rung waren vielerlei Geschehnisse voraus- gegangen,
von denen nun berichtet werden soll. Ein jeder B{\"u}rger der Scherbe
hat sicherlich von der brutalen Zerst{\"o}rung der Metropole Tripolis
Nova durch die damals fast unbekannte Nation "`Harsh Judgement"'
geh{\"o}rt.\\\\Eine Welle des Entsetzens und der Entr{\"u}stung durchlief
damals die diplomatische Welt der Scherbe, was zu einem breiten
B{\"u}ndniss gegen die Terrornation f{\"u}hrte, mit dem Ziel, deren Stadt,
Roma, zu zerst{\"o}ren. Angef{\"u}hrt wurde dies B{\"u}ndniss von der Nation
"`Stimmen des Sturms"', in deren Besitz die zerst{\"o}rte Stadt
Tripolis Nova gelegen hatte. Lange Zeit wurde die Stadt von den
vereinten Truppen des B{\"u}ndnisses belagert, bis vor einigen Wochen
Nachtschatten, Anf{\"u}hrerin der Nation Harsh Judgement und
B{\"u}rgermeisterin Romas, "`das Ende des Sturms"', die Kapitulation,
bekanntgab. Die Herrschaft {\"u}ber Roma wurde an anGur weitergegeben,
der zu dieser Zeit Mitglied der Nation "`Reiter der Apokalypse"'
war. Die Stadt wurde kurz darauf in "Apokalyptica" umbenannt, ein
erster Schlusstrich war gezogen, die Kapitulation Harsh Judgements
ausgesprochen, der Krieg f{\"u}r die Stimmen des Sturms gewonnen und
erledigt, will man meinen. Doch war dies dennoch nicht das Ende.
Da auch die Reiter der Apocalypse einen hoch zweifelhaften Ruf
genie{\ss}en und bereits zuvor Storm Peak angegriffen hatten, wurde
die Belagerung der Stadt weitergef{\"u}hrt, trotz neuem Herrscher und
neuem Namen. Mit Grund daf{\"u}r mag sein, dass beiden Nationen
Verbindungen zum K{\"o}nigreich Bearpaw nachgesagt wurden und sich
K{\"o}nig Zaracas die Vorstellung aufdr{\"a}ngen mochte, dass lediglich
Namen, nicht aber Absichten, gewechselt worden waren. Nun aber,
vor wenigen Tagen, wurde auch unter diese Angelegenheit ein
Schlusstrich gezogen, als K{\"o}nig Zaracas h{\"o}chstselbst das Rathaus
Apocalypticas den Flammen zum Fra{\ss}e gab. Nicht lang darauf wurde,
nicht unweit von der Stelle, da nun Romas Ruinen rauchen, eine
neue Stadt namens Metropolis errichtet; vom Lehen eines Reiters
der Apocalypse. Freie Wahlen wurden jedoch bereits zugesichert, so
dass man hoffen kann, dass dieser zweite Schlussstrich tats{\"a}chlich
der endg{\"u}ltige ist. Dies nun waren die Fakten in dieser
Angelegenheit; Spekulationen gibt es, wie bereits angedeutet,
viele mehr. Doch niemand, au{\ss}er den Beteiligten vermag solche
Spekulationen vollst{\"a}ndig zu verifizieren oder falsifizieren.
Dennoch will ich die "popul{\"a}rsten" Theorien an dieser Stelle
erw{\"a}hnen, denn auch das Wissen darum mag von Nutzen sein, auch
wenn {\"u}ber den Wahrheitsgehalt Unklarheit herrschen mag. So sprach
man schon fr{\"u}h davon, dass Harsh Judgement von K{\"o}nig Walnut f{\"u}r
ihre grausige Tat bezahlt worden w{\"a}ren, ebenso, dass K{\"o}nig Walnut
die Region um Roma zum Ausheben seiner Armeen nutzen w{\"u}rde.
W{\"a}hrend des Feldzuges gegen Harsh Judgement wurden Stim- men laut,
die dem K{\"o}nig vorwarfen, Geb{\"a}ude der Terrornation unter seinen
Schutz zu stellen. Als letztendlich Nachtschatten kapitulierte,
mag die Vermutung nahe liegen, dass sie hierzu von K{\"o}nig Walnut
gezwungen worden w{\"a}re, auf dass er das Gebiet um Roma weiter- hin
als Aushebungsgebiet f{\"u}r seine Armeen, die gegen die
Tuathen-Nationen ziehen sollten, nutzen k{\"o}nne. Wie bereits erw{\"a}hnt
sagt man auch den Reitern der Apocalypse, die nach Nachtschatten
die Herrschaft {\"u}ber die Stadt {\"u}bernahmen, eine enge Verbindung zum
K{\"o}nigreich Bearpaw nach. Als jedoch die Stimmen des Sturms nicht
nachlie{\ss}en, Apocalyptica zu belagern, lie{\ss} K{\"o}nig Walnut Armeen gen
Casa Nova ziehen, um dort Raum zu finden, auf dem er neue Armeen
ausheben k{\"o}nnte. Wie gesagt sind dies lediglich Spekulationen und
Theorien, keineswegs Fakten, dennoch mag hinter diesem Gespinst
von Vermutungen ein wahrer Kern liegen, wo jedoch, das verm{\"o}gen
uns wohl nur wenige zu sagen. Mehr Informationen
\chapter{Casa Nova}Wie bereits im vorherigen Artikel angedeutet, gibt es zur Zeit
auch um Casa Nova, das einst Atreides Monument hie{\ss}, viel
Diskussion. K{\"o}nig Walnut hatte die Belagerung der Stadt begonnen,
als sich die B{\"u}rgermeisterin der Stadt, die Pl{\"u}nderin Polarstern
Tshutsho im Krieg auf die Seite der Tuathen stellte. Weiterhin war
die Stadt Heimat des ber{\"u}chtigten Freibeuters R{\"u}diger van de
Schnee- klokjes gewesen. Um ihre Stadt vor dem Untergang zu retten
legte jedoch Tshutsho das Amt der B{\"u}rgermeisterin nieder und
{\"u}bergab es an Highfly, Mitglied der Union freier B{\"u}rger Gaias
(UfGB).\\\\Die Belagerung der recht wehrlosen Stadt wurde jedoch
nichtsdestotrotz fortgesetzt. Gr{\"u}nde hierf{\"u}r waren laut K{\"o}nig
Walnut Geb{\"a}ude einiger Pl{\"u}nderer, die innerhalb der Stadt stehen,
so vor allem von besagtem R{\"u}diger van de Schneeklokjes. Viele aber
vermuteten hinter der Belagerung der Stadt die Bestrebung des
K{\"o}nigreiches, aufgrund des bevorstehenden Verlustes
Roma/Apocalypticas sich an dieser Stelle, die nicht weit von den
Ruinen Apocalypticas entfernt ist, neue M{\"o}glichkeiten zur
Truppenaushebung zu beschaffen. So wurde auch auf dem
diplomatischen Wege keinerlei Einigung {\"u}ber eine Aufhebung der
Belagerung erzielt, da die UfGB nicht bereit war, eine hohe
Millionensumme an den K{\"o}nig zu zahlen. Highfly, gl{\"u}ckloser
B{\"u}rgermeister der belagerten Stadt, {\"u}bergab bald schon sein Amt an
Mystica, Oberste des Clan of Lions. Trotz aller Bem{\"u}hungen wurde
die Belagerung der Stadt dennoch fortgesetzt, ja sie dauert bis
zum heutigen Tage an. Mittlerweile hat auch Mystica ihr Amt
abgegeben, da sie keinen Ausweg aus der Situation sah. Vor allem
die Bev{\"o}lkerung Casa Novas hat stark unter der Belagerung
gelitten, ist doch ihre Zahl von {\"u}ber 100.000 auf kaum mehr als
drei Dutzend gefallen. Aufruhr verursachte auch die Requirierung
des Landes rings um Casa Nova durch das K{\"o}nigreich Bearpaw, wie
auf dieser Karte zu sehen ist. {\"U}ber den urspr{\"u}nglichen Besitzer
dieser L{\"a}ndereien jedoch gehen die Meinungen auseinander. W{\"a}hrend
die H{\"a}mmer der Ehre behaupten dies sei ihr Land gewesen, so
behauptet die UfBG das Land um Casa Nova w{\"a}re ihnen von den
H{\"a}mmern der Ehre zugesprochen worden. K{\"o}nig Walnut lie{\ss} verlauten,
dass dies Land Gebiet der Tuatha de Taranis war (erkennbar an den
verbliebenen Feldern links oben und rechts unten), womit die
Requirierung durch den Kriegszustand zwischen beiden Nationen
gerechtfertigt w{\"a}re. Wieder ist die Wahrheit wohl nur denen
bekannt, die direkt am Geschehen beteiligt waren.
\chapter{F{\"a}lle von Barbarei}
An dieser Stelle werden fortan jegliche Missetaten, bei denen
Zauberer von Loh zu Schade kommen aufgelistet werden, auf dass ein
jeder Zauberer von den {\"U}belt{\"a}tern wei{\ss} und gegen sie agieren kann.
In blau geschriebene Namen sind die von Wiederholungst{\"a}tern, die
bereits zuvor ZvL-G{\"u}ter gepl{\"u}ndert haben.\\\\
Tevion Mc Leod :{\"U}berf{\"a}lle auf Karawanen von Corintur, Der Ewigen,
Shi Long, Krabbe, Angriff auf Geb{\"a}ude Camarons und Armee von
Krabbe\\Harald : {\"U}berfall auf Karawane von Adalbert von Askanien
BlackDragon : {\"U}berfall auf Karawane von Opa Wetterwachs\\Arturs :
{\"U}berfall auf Karawane von Adalbert von Askanien\\ScriptorMagus :
{\"U}berf{\"a}lle auf Karawanen von Opa und Adalbert \\OgReXy : {\"U}berfall
auf Karawane von Opa Wetterwachs.\\Blackbeard : {\"U}berfall auf
Karawane von Claudius XI\\Imeriath Bangoral : Geb{\"a}udepl{\"u}nderung
bei Adalbert von Askanien\\Sith : {\"U}berfall auf Armee von Anton,
Geb{\"a}udepl{\"u}nderungen bei garm, Gamling (Almaren)\\DOGMA : {\"U}berfall
auf Karawane von Opa Wetterwachs\\Titan : {\"U}berfall auf Karawane
von Opa Wetterwachs\\Elamshina Asdana : Pl{\"u}nderung eines Geb{\"a}udes
von Opa Wetterwachs\\Ahab : {\"U}berfall auf Karawane von Davatar
\\Squatina : {\"U}berfall auf Armee von Anton Fugger, Angriff auf
Geb{\"a}ude von Claudius XI\\Talasin : {\"U}berfall auf Armee von Krabbe,
offener Angriff auf Claudius XI\\Aluvian : {\"U}berfall auf Karawane
des Ewigen\\Gnarf Dracula : {\"U}berfall auf Geb{\"a}ude von Thyrion\\
Borbarad : {\"U}berfall auf Karawane von Opa Wetterwachs\\Cedric of
Lancaster : {\"U}berfall auf Karawane von Opa Wetterwachs\\Magier des
Schwarzen Chaos : {\"U}berfall auf Karawane des Nathaniel\\Salmakia
von Svalbard : {\"U}berfall auf Karawane des Nathaniel Thorbart :
{\"U}berfall auf Karawane des Nathaniel\\Dagobert : Angriff auf garm
und Atlan1 (Almaren)\\Nachtschatten : {\"U}berfall auf Karawane von
Chicola\\BurningSoul : {\"U}berf{\"a}lle auf Karawanen von Chicola und
Nathaniel\\AlexRamone : {\"U}berfall auf Geb{\"a}ude von Chicola \\Smutje
Hefekoss : {\"U}berfall auf Karawane von Claudius XI Scarabeus :
{\"U}berfall auf Geb{\"a}ude von Claudius, Angriff auf Corintur
Arkhantorin \\MacEwen : Angriff auf Geb{\"a}ude von claudius XI
\\-L{\"o}we- : Angriffe auf Geb{\"a}ude von Claudius XI und diverser
Almaren

\chapter{Wipe!}
Der Zeitpunkt, von manchen lange ersehnt, von manchen gef{\"u}rchtet,
ist da: der Wipe kommt! Dies hat Aronius am Mittwoch im
allgemeinen Forum angek{\"u}ndigt. In zwei Monaten soll endlich die
Beta-Phase abgeschlossen sein und die erste "richtige" Runde SW
beginnen. Kostenlos wird dies jedoch nicht sein. Laut Aronius wird
man einen monatlichen Obulus zu entrichten haben, dessen H{\"o}he sich
wahrscheinlich irgendwo zwischen 3,50 bis 5� befinden wird. Die
Zahlungen sollen {\"u}ber 4Players erfolgen, die daf{\"u}r bereits ein
System entwickelt haben, n{\"a}heres hierzu gibt es zu einem sp{\"a}teren
Zeitpunkt. Bis zum Wipe sollen s{\"a}mtliche Neuerungen am Spiel
eingef{\"u}hrt worden sein, so dass w{\"a}hrend der Alpha-Phase keine
weiteren Wipes oder gro{\ss}en {\"A}nderungen im Spiel mehr vorgenommen
werden m{\"u}ssen.\\\\Es wird eine komplett neue Karte geben, in der
alle Klimazonen und Rohstoffe integriert
 sind und auf der auch f{\"u}r jeden Spieler mehr Platz vorhanden sein soll.
Der Neuanfang soll allgemein schwierigier sein als w{\"a}hrend der
Beta-Phase und das Spiel allgemein noch langfristiger ausgelegt -
wer also damit rechnet bereits nach einer Woche bereits beschw{\"o}ren
zu k{\"o}nnen, hat sich geschnitten, zudem will Aronius f{\"u}r sogenannte
"`High-End"'-Charaktere mehr Handicaps einbauen, wie auch bereits
schon geschehen. W{\"a}hrend sich bereits im RPG-Forum die ersten
Endzeit-Propheten einfinden, w{\"a}re es f{\"u}r die Reorganisation der
Nation nach dem Wipe nicht unwichtig, wenn diejenigen, die jetzt
schon sicher wissen, dass sie auch nach dem Wipe dabei (und bei
den ZvL) sein werden, sich im Forum melden w{\"u}rden. Euch allen viel
Erfolg nach dem Wipe und lasst's davor nochmal krachen ;)

\chapter{RL-Treffen} Anton hat im Forum den Vorschlag gemacht, dass man mal ein
RL-Treffen der ZvL organisieren k{\"o}nnte. Das Ganze befindet sich -
noch! - in einer sehr groben Planungsphase, doch jede Beteiligung
ist trotzdem erw{\"u}nscht und gewollt. Vor allem muss erstmal ein Ort
gefunden werden, der f{\"u}r alle annehmbar ist. Also wenn jemand
zentral oder vielleicht eher etwas im S{\"u}den von Deutschland wohnt
und ihm irgendwelche R{\"a}umlichkeiten zur Verf{\"u}gung stehen, m{\"o}ge der
sich bitte melden. Da wir recht viele Schweizer in der Nation
haben, m{\"u}ssen wir uns mehr s{\"u}dlich orientieren, damit auch die die
M{\"o}glichkeit haben, an dem Treffen teilzunehmen. W{\"a}re cool wenn wir
es hinkriegen w{\"u}rden, das tats{\"a}chlich statt- finden zu lassen :)


\end{document}
