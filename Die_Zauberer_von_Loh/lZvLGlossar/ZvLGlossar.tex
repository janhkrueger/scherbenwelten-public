%-------------------------------------------------------------------------------------------------------------------------
% Dokument f"ur die ZvL-Geschichte "uber die Wiederentdekcung des San-Titels im SW-Jahre 415.
% Hintergrund ist das die Zauberer von Loh im Jahre 415 ihre eigentlichen Wurzeln nicht mehr pr"asent haben
% und einen Weg suchen ihre alten Traditionen wiedre zu finden.
%
% Autoren: Jan H. Kr"uger, Andre Engelbertz
%
% Erstellt: 20.04.2004
%-------------------------------------------------------------------------------------------------------------------------
% "Anderungen:
% 20.04.2004: Erstellung und grobes Layout
% 21.04.2004: Fortf"uhrung der Geschichte mit Hauptaugenmerk auf die Geschehnisse in der Vergangenheit.
%             "Anderung am Layout, Einf"ugen von FancyHeadern sowie dem ZvL und Olimanir-Logo
% 22.04.2004: Fortf"uhrung, Einbinden der Geschichte um \tiername. Einf"ugen des Glossars.
%
% Geplante "Anderungen: Fortf"uhrung und Abschluss der Geschichte.
%-------------------------------------------------------------------------------------------------------------------------
 

\documentclass[11pt, twocolumn, a4paper, titlepage]{book}
\usepackage{ngerman, fancyhdr, graphicx}
\setlength{\headheight}{14.0pt}
\pagenumbering{arabic}
\setlength{\parindent}{0pt}
\newcommand{\Charname}{Pummelwurst }
\newcommand{\camaronposition}{Camaron }
\newcommand{\Kurzname}{Flitwig }
\newcommand{\samm}{San Achanjati }
\newcommand{\zvl}{Zauberer von Loh }
\newcommand{\tiername}{Vertat }
\newcommand{\andremail}{andre.engelbertz@gmx.de}
\newcommand{\mail}{insulae@janhkrueger.de}

\hyphenation {Weges-rand Schauf-fel-zau-ber Phe-ron Zau-ber Ma-gie Mi-no-tau-ren Pum-mel-wurst}

\pagestyle{fancy}



\begin{document}

\fancyhead[L]
{ \LARGE{Glossar} }

\fancyhead[C]
{ }

\fancyhead[R]
{ Stand: \today }

\fancyfoot[l]
{ %Einf"ugen des Olimanirwappens
\begin{picture}(0,0)
      \put(455,-40)
      {
         \includegraphics[scale=0.40]{olimanir.png}
      }
\end{picture}
%Einf"ugen des ZvL-Wappens
\begin{picture}(0,0)
      \put(285,-30)
      {
         \includegraphics[scale=0.40]{ZvL.jpg}
      }
\end{picture}
}


\fancyfoot[r]
{ \small{http://www.die-zauberer-von.loh.de/forum} }

\title{ \Huge{Story {\"u}ber die Wiederentdeckung der Sans}}

\author{
    Jan H. Kr{\"u}ger\thanks{\mail}
\and
    Andre Engelbertz\thanks{\andremail}
}
\date{Bad Homburg den \today}

%Glossar mit wichtigen Begriffen
\begin{description}
\item[Clepsydra]Vergleichbar mit einer Sanduhr, einziges Zeitmessger�t.
\item[Dom]Freundschaftliche Anrede. Gleichzusetzen mit Kumpel.
\item[Dopa]Ein feuriger Schnaps, der einen in k"ampferische Extase treibt.
\item[Eisgletscher von Sicce]Dorthin kommt die Seele wenn man stirbt.
\item[Jikai]Bezeichnung f"ur einen schwierigen Kampf.
\item[Jikaida]bevorzugtes,sehr beliebtes Brettspiel.
\item[Kapt]Bezeichnung f"ur einen Armeegeneral.
\item[Ib]Geist und Seele.
\item[Lahal]Offizielle Begr"u�ungsanrede unter den Zauberern von Loh. F"ur Freunde gedacht im Unterschied zu Llahal.
\item[Leem]Verabscheuungsw"urdige Gottheit der Kinderopfer gebracht werden.
\item[Llahal]Formelle Begr"u�ungsandre f�r Fremde.
\item[Llanitsch]Halt!
\item[Loh]Tropfenf"ormiger Heimatkontinent der Zauberer von Loh.
\item[Lupu]Zustand in den ein Zauberer sich versetzt um an einem anderen Ort etwas auszuspionieren oder jemandem etwas mitzuteilen.
\item[Luz]Grosse rote Sonne.
\item[Lynxter]Lohisches Schwert.
\item[Paktun]Anderes Wort f"ur S"oldner.
\item[Remberee]Abschiedsfloskel welche von den Bewohnern von Loh benutzt wird. Vergleichbar mit Auf Wiedersehen.
\item[San]Titel der f"ur Weiser/Gelehrter steht und Zauberern/Priestern verliehen wird.
\item[Sazz]Alkoholfreies Getr"ank das gut den Durst l"oscht.
\item[Walfarg]Grosses Reich von dem einst Loh den Mittelpunkt bildete.
\item[Walig]kleine gr"une Sonne.
\item[ZvL]Abk"urzung f�r Zauberer von Loh.
\end{description}

\end{document}

