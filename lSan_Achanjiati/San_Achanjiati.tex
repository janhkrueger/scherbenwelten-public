\documentclass[10pt, a4paper]{proc}
\usepackage{ngerman, fancyhdr}
\setlength{\headheight}{14.0pt}
\newcommand{\Charname}{San Achanjiati Mentem Mortis re Os }
\newcommand{\Charnametitel}{San Achanjiati }
\newcommand{\Kurznametitel}{San Achanjiati }
\newcommand{\Kurzname}{Achanjiati }
 
\hyphenation {Achan-jiati }
\markboth{Charakterbeschreibung}{\large {\Charname}}
\pagestyle{fancy} \setlength{\parindent}{0pt}
\title{ \Huge{\Charname }}
\author{Jan H. Kr{\"u}ger\\game.insulae@googlemail.com}
\date{Bad Homburg den \today}
\begin{document}
\maketitle
%\fontfamily{cmss} \selectfont
Nicht viele Bewohner der Scherbe kennen ihn noch, diesen Magier. Er wandelte bereites mit den Altvorderen auf der ersten Scherbe und lernte
 so die Welt und ihre Prinzipien kennen. Gegen Ende der ersten Scherbe  - auf der er damals noch nur als "`Achanjiati"` bekannt war - trat er
 dann einer gro�en und m�chtigen Zauberernation bei, den "`Zauberer von Loh"`. Dort wurde ihm auch der Titel "`San"` verliehen. Dann erlosch die
  erste Scherbe und er wechselte hin�ber in die neue, zweite Scherbe.

Auch ier wieder schloss er sich den "Zauberer von Loh" an und erkundete die Scherbe von neuem, beendete seine Wanderschaft allerdings schon
sehr fr�h um ein Gasthaus neben zwei Trollbr�cken aufzubauen und zu betreiben.
Nach einiger Zeit zog er sich dann immer mehr zur�ck von dem Alltagsleben und betrieb Studien wie er sagte.
Offenbar waren es dese Studien die es ihm erm�glichten w�hrend der gro�en Seuche unbeschadet selbst mit Infizierten zu interagieren...
Wendarias Hauch ben�tigte er nie.

W�hrend des D�monenkrieges dann gelang ihm ein Durchbruch bei seinen Forschungen. Die Erschaffung untoten Lebens bei beibehaltung der geistigen
 F�higkeiten, des Wissens und des Wesens des verwandelten Objektes. Er erschuff intelligentes untotes Leben wie es die Scherbe bisher noch
 nicht kannte.
Daraufhin - aber auch wegen seiner heimlichen Verehrung zu ihr und weil sie einen Sieg der Scherbenretter vorraussah und um ihre Existenz
f�rchtete offenbarte sich San Achanjiati die Herrin der Untoten: Olimanir.
Im Austausch f�r seine Hilfe bot sie ihm Wissen, Macht und Unsterblichkeit. Er willigte ein.

Mit Olimanirs Hilfe vertiefte er seine Studien und verfeinerte die nekromantischen Rituale. Als beide - der Magier und die dunkle G�ttin - zufrieden waren begann die zweite Phase des Planes. Dazu musste er allerdings ungest�rt sein und fernab von neugierigen Blicken.
Er brach deshalb alles ab was er bei Trollbruck besa� und siedelte sich auf der Hauptinsel seiner Nation an.
Dort an einer abgelegenen Stelle bereitete er alles weitere vor. Ein Ge�� zu erschaffen in dem die Essenz Olimanirs erhalten und gespeichert bliebe selbst wenn ihre normale H�lle wie die der anderen Schreckensg�tter zerst�rt werden sollte.

Nach ein paar Wochen dann war das gro�e Ritual fertig und alles war mittels einer Handvoll durch Maige und Olimanirs Macht willenlos gemachter Helfer in einem Tunnel- und Katakombensystem unterhalb einer von San Achanjiati errichteten Jagdh�tte vorbereitet. Am Schluss des Rituals dann fuhr Olimanir in einen pr�parierten Sch�del eines Babydrachen und durchblickt die Augen nun durch funkelnde Augen aus roten Rubinen.

Vor der endg�ltigen Vernichtung bewahrt machte sich Olimanir nun daran ihren Teil der Abmachung einzuhalten und gab San Achanjiati weitere Anweisungen. Besondere und seltene Rohstoffe wurden herangeschafft, seltsame Ger�te gebaut.
Dann vollzog er sich einer magischen Prozedur aus der er von Grund auf ver�ndert wieder hervorkam: Kein Fetzen Fleisch hing mehr an seinen Knochen, er war untot ohne gestorben zu sein. Ein Wort f�r das was er wurde gibt es noch nicht und so behielt er seinen alten Namen bei.

San Achanjiati sieht nach seiner Umwandlung v�llig anders aus. Er ist v�llig ohne Fleisch und bsteht nur noch aus seinen Knochen welche allesamt mit einer d�nnen Schicht reinen Enduriums �berzogen wurden. Normal betrachtet sieht diese Schicht v�llig glatt und eben aus, mit Adleraugen betrachtet w�rden sich jedoch feine Muster Adern oder einem Blattgerippe gleich  hervortun. In seinen Augenh�hlen funkeln zwei gro�e, zu Augen geschliffene schwarze Edelsteine.
Durch das v�llige Fehlen von Muskeln und anderem Gewebe sieht er nun sehr d�rr aus, trotz des weiten Umhanges den er tr�gt. Wenn er mit Lebenden zusammen ist tr�gt er meist schwarze Lederhandschuhe welche seine Knochenh�nde verbergen. Sein gesicht ist dann stehts in den Tiefen seiner Kapuze verh�llt. Er ist unsterblich geworden, Alter, Krankheit und Hunger k�nnen ihm nichts mehr anhaben.

Seit dem Tag seiner Umwandlung nennt er sich nun "`San Achanjiati Mentem Mortis re Os"`, erster Hohepriester Olimanirs und erster Nekromant der Scherbe, der Geist von Tod und Knochen.
\end{document}

